\documentclass{beamer}

\usepackage{graphicx}
\usepackage{hyperref}
\usepackage[english]{babel}
\usepackage[utf8]{inputenc}

\usepackage[T1]{fontenc}
\usepackage{lmodern}
\usepackage[bitstream-charter]{mathdesign}

\usepackage{booktabs}
\usepackage{amssymb}
\usepackage{xcolor}
\usepackage{siunitx}
\sisetup{exponent-product = \cdot}

\usepackage{feynman}

\mode<presentation>
{%
  \usetheme{m4ck}
  \setcounter{showProgressBar}{1}
  \setcounter{showSlideNumbers}{1}
}


% _____________________________________________________________________________
% meta data                                                           META DATA

\title[Short Title]{This is a long title}
\subtitle{Well, this is a subtitle}
\author{Christoph Weinsheimer}
\institute[Uni Mainz]
{%
    Johannes Gutenberg-University Mainz
}
\date{Spring 2015}



% _____________________________________________________________________________
% title page                                                         TITLE PAGE

\begin{document}
\maketitle

% _____________________________________________________________________________
% slides                                                                 SLIDES

\section{Example}

\begin{frame}{Example Frame}
  \begin{columns}
    \column{0.5\textwidth}
      \begin{block}{Example Block}
        \begin{wideitemize}
          \item Bullet Point 1
            \begin{itemize}
              \item And some sub item
            \end{itemize}
          \item another one
        \end{wideitemize}
        \begin{enumerate}
          \item Enumeration
            \begin{enumerate}
              \item Subenumeration
            \end{enumerate}
        \end{enumerate}
        \begin{description}
            \item[Key:] Value
        \end{description}
      \end{block}
    \column{0.5\textwidth}
      \begin{figure}[ht]
        \centering
        \shadowimage[width=0.9\textwidth]{JGU.pdf}
        \caption{Logo of the university}
      \end{figure}
  \end{columns}
\end{frame}



\begin{frame}{Another Slide}
  \begin{center}
    \[
      \sum_i^\infty q^k = \frac{1}{1-q}
    \]
  \end{center}
\end{frame}



\backupbegin%
\begin{frame}{Some backup slide}
  \begin{center}
    This is the first backup slide
  \end{center}
\end{frame}
\backupend%


\end{document}
