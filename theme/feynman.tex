\usepackage{amsmath,tikz}

% FEYNMAN TIKZ 
\usetikzlibrary{arrows}
\usetikzlibrary{positioning}               % For "above of=" commands
\usetikzlibrary{calc,through}              % For coordinates
\usetikzlibrary{decorations.pathreplacing} % For curly braces
\usetikzlibrary{decorations.pathmorphing}  % For Feynman Diagrams
\usetikzlibrary{decorations.markings}

% style definition
\tikzset{%
          >=latex', %%  Uncomment for more conventional arrows
    vertex/.style={circle,fill,inner sep=0mm, minimum size=4pt},
    blob/.style  ={circle,draw,inner sep=0mm, minimum size=30pt},
    vector/.style={decorate, decoration={snake}, draw},
    provector/.style={decorate, decoration={snake,amplitude=2.5pt}, draw},
    antivector/.style={decorate, decoration={snake,amplitude=-2.5pt}, draw},
    fermion/.style={draw=black, postaction={decorate},
        decoration={markings,mark=at position 0.60 with {\arrow[draw=black]{>}}}},
    fermionbar/.style={draw=black, postaction={decorate},
        decoration={markings,mark=at position 0.60 with {\arrow[draw=black]{<}}}},
    fermionnoarrow/.style={draw=black},
    gluon/.style={decorate, draw=black,
        decoration={coil,amplitude=4pt, segment length=5pt}},
    scalar/.style={dashed,draw=black, postaction={decorate},
        decoration={markings,mark=at position 0.60 with {\arrow[draw=black]{>}}}},
    scalarbar/.style={dashed,draw=black, postaction={decorate},
        decoration={markings,mark=at position 0.60 with {\arrow[draw=black]{<}}}},
    scalarnoarrow/.style={dashed,draw=black},
    electron/.style={draw=black, postaction={decorate},
        decoration={markings,mark=at position 0.60 with {\arrow[draw=black]{>}}}},
    bigvector/.style={decorate, decoration={snake,amplitude=4pt}, draw},
}

%%%%%%%%%%%%%%%%%%%%%%%%%%%%%%%%%%%%%%%%%%%%%%%%%%%%%%%%%%%%%%%%%%%%%%%%%%%%%%%
% EXAMPLES
%

%             \begin{tikzpicture}[line width=1.5 pt, scale=1.5]
%                 \node[vertex]     (v1) at (0,0) {}; % umistnit vertexi na platne
%                 \node[vertex]     (v2) at (1,0) {};
%                 \draw[fermion]    (v1) -- ++(140:1); % vstupni nozicka (trosku hack protoze vychazi z v1 a ne vstupuje
%                 \draw[fermionbar] (v1) -- ++(220:1);
%                 \draw[vector]     (v1) -- (v2); % intermedialni bozon z jednoho vertexu do druheho
%                 \draw[fermion]    (v2) -- ++(40:1); % vystupni nozicka
%                 \draw[fermionbar] (v2) -- ++(-40:1);
%                 \node at (0.5,0.5) {$Z/\gamma/W$};
%                 \node at (0.5,1) {d\v rel Jan a dablj\r u};
%             \end{tikzpicture}
%             \begin{tikzpicture}[line width=1.5 pt, scale=1.5]
%                 \node[vertex]     (v1) at (0,0) {};
%                 \node[vertex]     (v2) at (1,0) {};
%                 \draw[fermion]    (v1) -- ++(140:1);
%                 \draw[fermionbar] (v1) -- ++(220:1);
%                 \draw[vector]     (v1) -- (v2);
%                 \draw[vector]     (v2) -- ++(40:1);
%                 \draw[vector]     (v2) -- ++(-40:1);
%                 \node at (0.5,0.5) {$Z/\gamma$};
%                 \node[right] at ($(v2)+(40:1)$) {$W$};
%                 \node[right] at ($(v2)+(-40:1)$) {$W$};
%                 \node at (0.5,1) {troj-cejchovac\'i vazba};
%             \end{tikzpicture}
%             \begin{tikzpicture}[line width=1.5 pt, scale=1.5]
%                 \node[vertex]     (v1) at (0,0) {};
%                 \node[vertex]     (v2) at (0,1) {};
%                 \draw[fermion]    (v1) -- ++(180:1);
%                 \draw[fermionbar] (v2) -- ++(180:1);
%                 \draw[fermion]    (v2) -- (v1);
%                 \draw[vector]     (v1) -- ++(0:1);
%                 \draw[vector]     (v2) -- ++(0:1);
%                 \node[left] at (-1,1) {$f$};
%                 \node[left] at (-1,0) {$\bar f$};
%                 \node[right] at (1,1) {$Z$};
%                 \node[right] at (1,0) {$W$};
%                 \node at (0.0,1.5) {\v cajov\'y kan\'al};
%             \end{tikzpicture}
%             \begin{tikzpicture}[line width=1.5 pt, scale=1.5]
%                 \node[vertex]     (v1) at (0,1) {};
%                 \node[vertex]     (v2) at (0,0) {};
%                 \draw[fermionbar] (v1) -- ++(180:1);
%                 \draw[fermion]    (v2) -- ++(180:1);
%                 \draw[fermion]    (v1) -- (v2);
%                 \draw[vector]     (v1) -- ++(-40:1.5);
%                 \draw[vector]     (v2) -- ++(40:1.5);
%                 \node[left] at (-1,1) {$f$};
%                 \node[left] at (-1,0) {$\bar f$};
%                 \node at (0.0,1.5) {ty kan\'ale!};
%             \end{tikzpicture}
%             \begin{tikzpicture}[line width=1.5 pt, scale=1.5]
%                 \node[vertex]     (v1) at (0,0.6) {};
%                 \node[vertex]     (v2) at (0,0) {};
%                 \node[vertex]     (v3) at (0.6,0) {};
%                 \node[vertex]     (v4) at (0.6,0.6) {};

%                 \draw[fermion]    (v1) -- (v2);
%                 \draw[fermion]    (v2) -- (v3);
%                 \draw[fermion]    (v3) -- (v4);
%                 \draw[fermion]    (v4) -- (v1);

%                 \draw[gluon]      (v1) -- ++(140:1);
%                 \draw[gluon]      (v2) -- ++(220:1);

%                 \draw[vector]     (v3) -- ++(-40:1);
%                 \draw[vector]     (v4) -- ++(40:1);

%                 \node[above] at ($(v3)+(-34:1)$) {$W$};
%                 \node[below] at ($(v4)+( 34:1)$) {$W$};

%                 \node[above] at ($(v2)+(214:1)$) {$g$};
%                 \node[below] at ($(v1)+(146:1)$) {$g$};
%             \end{tikzpicture}

%             \begin{tikzpicture}[line width=1.5 pt, scale=1.5]
%                 \node[vertex]     (v1) at (0,0.8) {};
%                 \node[vertex]     (v2) at (0,0.2) {};
%                 \node[vertex]     (v3) at (0.5,0.5) {};
%                 \node[vertex,color=red]     (v4) at (1.2,0.5) {};

%                 \draw[fermion]    (v1) -- (v2);
%                 \draw[fermion]    (v2) -- (v3);
%                 \draw[fermion]    (v3) -- (v1);
%                 \draw[vector]     (v3) -- (v4);

%                 \draw[gluon]      (v1) -- ++(120:1);
%                 \draw[gluon]      (v2) -- ++(240:1);

%                 \draw[vector]     (v4) -- ++(-60:1);
%                 \draw[vector]     (v4) -- ++(60:1);

%                 \node[above] at ($(v4)+(-52:1)$) {$W$};
%                 \node[below] at ($(v4)+( 52:1)$) {$W$};

%                 \node        at ($(v3)+(0.4,0.3)$) {$Z/\gamma$};

%                 \node[right] at ($(v4)+(0.3,0.0)$) {TGC vertex};

%                 \node[above] at ($(v2)+(232:1)$) {$g$};
%                 \node[below] at ($(v1)+(128:1)$) {$g$};
%             \end{tikzpicture}


%             % Honza si udelal ucet na battlenet a masti karty, takze celou dobu posloucham jenom `Caz' Dingo!!!!', 'I hope you like my invections' a 'WrrrgghlWrrggglgz'
%             % takze si dal kolu a zustal tady celou noc, nad ranem se asi popereme o misto na zidli.

%             % Jo jinak jak jsem ti rikal ze jsem prisel na to jak udelat ten dummy vector, tak to nefunguje nicmene jsem to udelal systemem BRUTEFORCE.
%             % Takze kazdy typ ma svoji dummy v Eventu a vectory maji 100 itemu.. zatim lepsi nez nic.

%             % udelal jsem novy styl blob, v obrazku si ho muze vybarvit jak chces. DPI byl trochu tezsi ale zvladl jsem to


%             \begin{tikzpicture}
%                 \node[blob,fill=black!20!white] (b1) at (-1, 1.5) {};
%                 \node[blob,fill=black!20!white] (b2) at (-1,-1.5) {};
%                 \node[blob,fill=black!20!white] (b3) at ( 1, 0  ) {};

%                 \coordinate (o1) at ($(b1)+(180:1.5)$);
%                 \coordinate (o2) at ($(b2)+(180:1.5)$);
%                 \coordinate (o3) at ($(b3)+( 30:1.5)$);
%                 \coordinate (o4) at ($(b3)+(-30:1.5)$);

%                 \draw[fermionnoarrow, very thick] (o1) -- (b1);
%                 \draw[fermionnoarrow]             (b1) -- ++( 15:1.2);
%                 \draw[fermionnoarrow]             (b1) -- ++(  0:1.2);
%                 \draw[fermionnoarrow]             (b1) -- ++(-15:1.2);

%                 \draw[fermionnoarrow, very thick] (o2) -- (b2);
%                 \draw[fermionnoarrow]             (b2) -- ++( 15:1.2);
%                 \draw[fermionnoarrow]             (b2) -- ++(  0:1.2);
%                 \draw[fermionnoarrow]             (b2) -- ++(-15:1.2);

%                 \draw[fermionnoarrow] (b1) -- (b3);
%                 \draw[fermionnoarrow] (b2) -- (b3);

%                 \draw[fermion       ] (b3) -- (o3);
%                 \draw[fermion       ] (b3) -- (o4);

%                 \node[below] at (o1) {p};
%                 \node[above] at (o2) {p};
%                 \node at (0, 0.5) {i};
%                 \node at (0,-0.5) {j};
%                 \node[above] at (o3) {a};
%                 \node[below] at (o4) {b};

%             \end{tikzpicture}

%             \begin{tikzpicture}
%                 %blobs
%                 \node[blob,fill=black!20!white] (b1) at (-1, 1.5) {};
%                 \node[blob,fill=black!20!white] (b2) at (-1,-1.5) {};
%                 \node[blob,fill=black!20!white] (b3) at ( 1, 0.7) {};
%                 \node[blob,fill=black!20!white] (b4) at ( 1,-0.7) {};

%                 %outer
%                 \coordinate (o1) at ($(b1)+(180:1.5)$);
%                 \coordinate (o2) at ($(b2)+(180:1.5)$);
%                 \coordinate (o3) at ($(b3)+( 20:1.5)$);
%                 \coordinate (o4) at ($(b3)+(-20:1.5)$);
%                 \coordinate (o5) at ($(b4)+( 20:1.5)$);
%                 \coordinate (o6) at ($(b4)+(-20:1.5)$);

%                 %inner
%                 \coordinate (i1) at ($ (b1)!.5!(b3) $);
%                 \coordinate (i2) at ($ (b2)!.5!(b3) $);
%                 \coordinate (i3) at ($ (b1)!.5!(b4) $);
%                 \coordinate (i4) at ($ (b2)!.5!(b4) $);

%                 \draw[fermionnoarrow, very thick] (o1) -- (b1);
%                 \draw[fermionnoarrow]             (b1) -- ++( 35:1.2);
%                 \draw[fermionnoarrow]             (b1) -- ++( 20:1.2);
%                 \draw[fermionnoarrow]             (b1) -- ++(  5:1.2);

%                 \draw[fermionnoarrow, very thick] (o2) -- (b2);
%                 \draw[fermionnoarrow]             (b2) -- ++( -5:1.2);
%                 \draw[fermionnoarrow]             (b2) -- ++(-20:1.2);
%                 \draw[fermionnoarrow]             (b2) -- ++(-35:1.2);

%                 \draw[fermionnoarrow] (b1) -- (b3);
%                 \draw[fermionnoarrow] (b2) -- (b3);
%                 \draw[fermionnoarrow] (b1) -- (b4);
%                 \draw[fermionnoarrow] (b2) -- (b4);

%                 \draw[fermion       ] (b3) -- (o3);
%                 \draw[fermion       ] (b3) -- (o4);
%                 \draw[fermion       ] (b4) -- (o5);
%                 \draw[fermion       ] (b4) -- (o6);

%                 \node[below] at (o1) {p};
%                 \node[above] at (o2) {p};
                 
%                 \node[above] at (i1) {i};
%                 \node[above] at (i2) {j};
%                 \node[above] at (i3) {k};
%                 \node[above] at (i4) {l};

%                 \node[above] at (o3) {a};
%                 \node[above] at (o4) {b};
%                 \node[below] at (o5) {c};
%                 \node[below] at (o6) {d};

%             \end{tikzpicture}
